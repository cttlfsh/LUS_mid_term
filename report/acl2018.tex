%
% File acl2018.tex
%
%% Based on the style files for ACL-2017, with some changes, which were, in turn,
%% Based on the style files for ACL-2015, with some improvements
%%  taken from the NAACL-2016 style
%% Based on the style files for ACL-2014, which were, in turn,
%% based on ACL-2013, ACL-2012, ACL-2011, ACL-2010, ACL-IJCNLP-2009,
%% EACL-2009, IJCNLP-2008...
%% Based on the style files for EACL 2006 by 
%%e.agirre@ehu.es or Sergi.Balari@uab.es
%% and that of ACL 08 by Joakim Nivre and Noah Smith

\documentclass[11pt,a4paper]{article}
\usepackage[hyperref]{acl2018}
\usepackage{times}
\usepackage{latexsym}

\usepackage{url}

\aclfinalcopy % Uncomment this line for the final submission
%\def\aclpaperid{***} %  Enter the acl Paper ID here

%\setlength\titlebox{5cm}
% You can expand the titlebox if you need extra space
% to show all the authors. Please do not make the titlebox
% smaller than 5cm (the original size); we will check this
% in the camera-ready version and ask you to change it back.

\newcommand\BibTeX{B{\sc ib}\TeX}

\title{Language Understanding System: Mid-term Project}

\author{Montagner Andrea \\
  {\tt andrea.montagner@studenti.unitn.it}}

\date{\today}

\begin{document}
\maketitle
\begin{abstract}
The aim of this project is to develop o Spoken Language Understanding module for the Movie Domain capable of performing PoS tagging. In order to achieve this goal a language model is firstly created and then trained to predict unseen utterances from a testing pool of data. In the following sections multiple approaches with respective solutions are presented, each one evaluated in term of accuracy, precision, recall and f1 score. Finally these results are compared to each other in order to determine the best approach ti this problem.
\end{abstract}

\section{Introduction}

A Concept Tagger is a module which is capable of assigning a concept, in this case a tag, to words belonging to a sentence. This operation, though, requires the presence of a dataset composed by couples (words, tag) large enough in order to have a sufficient number of utterances to perform both the training and testing phase.\\
In particular, the material that was provided for this project was:
\begin{itemize}
	\item A train dataset, both for the Movie Domain and for the additional features,
	\item A test dataset, both for the Movie Domain and for the additional features,
	\item A Perl script, {\tt "conlleval.pl"}, to evaluate the performances in terms of accuracy, precision, recall and f1 score.
\end{itemize}


\section{Data Analysis}
\label{sec:datanalysis}

The provided dataset is the Microsoft NL-SPARQL dataset and it was already split into two parts: one for training and one for testing. In addition each of these groups of data was divided into main data for the Movie Domain and additional features.\\ 
To summarize, what was given were four files:
\begin{itemize}
	\item \textbf{NL-SPARQL.train.data} containing a two columns set of data, tab separated, where the first column represent the words and the second the tags,
	\item \textbf{NL-SPARQL.train.feats.txt} containing a three columns set of data, tab separated, where the first column represent the words and the second the tags and third the lemmas,
	\item \textbf{NL-SPARQL.test.data} containing a two columns set of data, tab separated, where the first column represent the words and the second the tags,
	\item \textbf{NL-SPARQL.test.feats.txt} containing a three columns set of data, tab separated, where the first column represent the words and the second the tags and third the lemmas.
\end{itemize}

In these files, each line contains a word (with the respective tag/lemma) of a sentence and at the end of each sentence there is an empty line as separator from the next.\\
Some examples are shown below:

\large\textbf{DATA EXAMPLE}

Analysis of the distributions, as far as the provided files are concerned, can be performed grouping for words as well as for tags. In the two following subsections, both analysis are shown.

\subsection{IOB Definition \& Distribution}
 
 Concept Tags are here represented using an IOB notation, which is a common tagging format when dealing with ...\\
 The used prefixes are:
 \begin{itemize}
 	\item \textbf{B} - Beginning of a span
 	\item \textbf{I} - Inside of a span
 	\item \textbf{E} - End of a span
 	\item \textbf{O} - Outside of a span
 \end{itemize}
 
As long as IOBs distributions is concerned, considering that the dataset is Movie Domain, the analysis mirrors the prediction and outputs as the most frequent tag {\tt movie.domain} divided in {\tt I.movie.domain} and {\tt B.movie.domain}.\\

In this analysis the tag {\tt O} is not taken into consideration because it can be considered a "stop-word" for tags.
 
\large\textbf{TABLE}

\subsection{Words Distribution}

Very similar are results for the analysis on the word frequency. Almost all of most frequent words are identified as stop words, like {\tt the}, {\tt of}, {\tt in}. Nonetheless there are exceptions, like {\tt movies} and {\tt movie}, which are relevant to understand the topic of the considered dataset.


\subsection{Additional Features}
\label{subsect:addfeat}



\begin{table}[t!]
\begin{center}
\begin{tabular}{|l|rl|}
\hline \bf Type of Text & \bf Font Size & \bf Style \\ \hline
paper title & 15 pt & bold \\
author names & 12 pt & bold \\
author affiliation & 12 pt & \\
the word ``Abstract'' & 12 pt & bold \\
section titles & 12 pt & bold \\
subsection titles & 11 pt & bold \\
document text & 11 pt  &\\
captions & 11 pt & \\
abstract text & 11 pt & \\
bibliography & 10 pt & \\
footnotes & 9 pt & \\
\hline
\end{tabular}
\end{center}
\caption{\label{font-table} Font guide. }
\end{table}

%Use 11 points for text and subsection headings, 12 points for section headings and 15 points for the title. 


\begin{table}
\centering
\small
\begin{tabular}{cc}
\begin{tabular}{|l|l|}
\hline
\textbf{Command} & \textbf{Output}\\\hline
\verb|{\"a}| & {\"a} \\
\verb|{\^e}| & {\^e} \\
\verb|{\`i}| & {\`i} \\ 
\verb|{\.I}| & {\.I} \\ 
\verb|{\o}| & {\o} \\
\verb|{\'u}| & {\'u}  \\ 
\verb|{\aa}| & {\aa}  \\\hline
\end{tabular} & 
\begin{tabular}{|l|l|}
\hline
\textbf{Command} & \textbf{ Output}\\\hline
\verb|{\c c}| & {\c c} \\ 
\verb|{\u g}| & {\u g} \\ 
\verb|{\l}| & {\l} \\ 
\verb|{\~n}| & {\~n} \\ 
\verb|{\H o}| & {\H o} \\ 
\verb|{\v r}| & {\v r} \\ 
\verb|{\ss}| & {\ss} \\\hline
\end{tabular}
\end{tabular}
\caption{Example commands for accented characters, to be used in, {\em e.g.}, \BibTeX\ names.}\label{tab:accents}
\end{table}

\begin{table*}
\centering
\begin{tabular}{lll}
  output & natbib & previous ACL style files\\
  \hline
  \citep{Gusfield:97} & \verb|\citep| & \verb|\cite| \\
  \citet{Gusfield:97} & \verb|\citet| & \verb|\newcite| \\
  \citeyearpar{Gusfield:97} & \verb|\citeyearpar| & \verb|\shortcite| \\
\end{tabular}
\caption{Citation commands supported by the style file.
  The citation style is based on the natbib package and
  supports all natbib citation commands.
  It also supports commands defined in previous ACL style files
  for compatibility.
  }
\end{table*}

\subsection{Figures and Tables}

%\textbf{Placement}: Place figures and tables in the
%paper near where they are first discussed if
%possible.  
%Wide figures and tables may run across both columns and should be placed at the top of a page.

\textbf{Placement}: Place figures and tables in the paper near where they are first discussed, as close as possible to the top of their respective column.

%Color illustrations are discouraged, unless you have verified that  
%they will be understandable when printed in black ink.

\textbf{Captions}: Provide a caption for every illustration; number each one
sequentially in the form:  ``Figure 1: Caption of the Figure.'' ``Table 1:
Caption of the Table.''  Type the captions of the figures and 
tables below the body, using the caption font size shown in Table~\ref{font-table}.

\section{Evaluation: Baseline method}

It is also advised to supplement non-English characters and terms
with appropriate transliterations and/or translations
since not all readers understand all such characters and terms.
Inline transliteration or translation can be represented in
the order of: original-form transliteration ``translation''.

\section{Other approaches: different training methods}
\label{sec:length}

The ACL 2018 main conference accepts submissions of long papers and
short papers.
 Long papers may consist of up to eight (8) pages of
content plus unlimited pages for references. Upon acceptance, final
versions of long papers will be given one additional page -- up to nine (9)
pages of content plus unlimited pages for references -- so that reviewers' comments
can be taken into account. Short papers may consist of up to four (4)
pages of content, plus unlimited pages for references. Upon
acceptance, short papers will be given five (5) pages in the
proceedings and unlimited pages for references. 

For both long and short papers, all illustrations and tables that are part
of the main text must be accommodated within these page limits, observing
the formatting instructions given in the present document. Supplementary
material in the form of appendices does not count towards the page limit.

However, note that supplementary material should be supplementary
(rather than central) to the paper, and that reviewers may ignore
supplementary material when reviewing the paper (see Appendix
\ref{sec:supplemental}). Papers that do not conform to the specified
length and formatting requirements are subject to be rejected without
review.

Workshop chairs may have different rules for allowed length and
whether supplemental material is welcome. As always, the respective
call for papers is the authoritative source.
\section{Error Analysis}
\section{Discussion}
\section*{Acknowledgments}

The acknowledgments should go immediately before the references.  Do not number the acknowledgments section ({\em i.e.}, use \verb|\section*| instead of \verb|\section|). Do not include this section when submitting your paper for review.

% include your own bib file like this:
%\bibliographystyle{acl}
%\bibliography{acl2018}
\bibliography{acl2018}
\bibliographystyle{acl_natbib}

\appendix

\section{Supplemental Material}
\label{sec:supplemental}
ACL 2018 also encourages the submission of supplementary material
to report preprocessing decisions, model parameters, and other details
necessary for the replication of the experiments reported in the 
paper. Seemingly small preprocessing decisions can sometimes make
a large difference in performance, so it is crucial to record such
decisions to precisely characterize state-of-the-art methods.

Nonetheless, supplementary material should be supplementary (rather
than central) to the paper. \textbf{Submissions that misuse the supplementary 
material may be rejected without review.}
Essentially, supplementary material may include explanations or details
of proofs or derivations that do not fit into the paper, lists of
features or feature templates, sample inputs and outputs for a system,
pseudo-code or source code, and data. (Source code and data should
be separate uploads, rather than part of the paper).

The paper should not rely on the supplementary material: while the paper
may refer to and cite the supplementary material and the supplementary material will be available to the
reviewers, they will not be asked to review the
supplementary material.

Appendices ({\em i.e.} supplementary material in the form of proofs, tables,
or pseudo-code) should come after the references, as shown here. Use
\verb|\appendix| before any appendix section to switch the section
numbering over to letters.

\section{Multiple Appendices}
\dots can be gotten by using more than one section. We hope you won't
need that.

\end{document}
